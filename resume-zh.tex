% !TEX TS-program = xelatex
%% Copyright (c) 2019-2020 Donghui Shen %%<saysin.github.io>
%% CC BY 4.0 License
%%
%% Created: 2019-10-11
%%

% Chinese version
\documentclass[zh]{resume}

% Adjust icon size (default: same size as the text)
\iconsize{\Large}

% File information shown at the footer of the last page
\fileinfo{%
  \faCopyright{} 2019--2020, Donghui Shen \hspace{0.5em}
  \creativecommons{by}{4.0} \hspace{0.5em}
  \githublink{saysin}{resume} \hspace{0.5em}
  \faEdit{} \today
}

\name{栋辉}{沈}

\keywords{Linux, Programming, Python, C, Shell}

\tagline{\icon{\faBinoculars}应聘专任教师岗位G类}%寻求岗位
%\tagline{博士研究生}%现在的身份或者职位

\photo{3.5cm}{求职.jpg}

\profile{
  \mobile{132-1610-7661}
  \email{shendonghui@zju.edu.cn}
  \github{saysin} \\
  \degree{物理学 \textbullet 博士}
  \university{浙江大学}
  \birthday{1992-12-01}
  \home{浙江 \textbullet 绍兴}
  % Custom information:
  % \icontext{<icon>}{<text>}
  % \iconlink{<icon>}{<link>}{<text>}
}

\begin{document}
\makeheader

%======================================================================
% Summary & Objectives
%======================================================================
{\onehalfspacing\hspace{2em}
物理学专业(光学方向)博士研究生,有扎实的物理、数学与编程基础,
擅长光束传输变换理论模拟和光学实验,主要研究领域为光学涡旋和光场调控。
热爱物理和计算机技术。熟练掌握 Shell、Python 和 C 语言编程。崇尚自由开源的精神,并一直在努力为开源世界贡献自己的力量。
\par}

%======================================================================

%======================================================================
\sectionTitle{教育背景}{\faGraduationCap}
%======================================================================
\begin{educations}
  \education%
    {2015.09}%
    [2020.06]%
    {浙江大学}%
    {理学院物理系}%
    {光学}%
    {博士(硕博连读)}

\separator{0.5ex}
\education%
{2011.09}%
[2015.06]%
{华侨大学}%
{信息科学与工程学院}%
{应用物理学}%
{学士}

\separator{0.5ex}
\education%
{2008.09}%
[2011.06]%
{春晖中学}%
{理科}%
{理科}%
{高中生}
\end{educations}

%======================================================================
\sectionTitle{科研成果}{\faAtom}
%======================================================================
\begin{itemize}
  \item 发表 2 篇第一作者SCI 论文,1篇第三作者SCI论文,另有一篇论文已提交至Annalen der Physik(物理二区)
  \\ 1.光学二区期刊\texttt{Optics Letters}——\link{https://doi.org/10.1364/OL.44.002334}{Measuring the topological charge of optical vortices with a twisting phase}
  \\ 2.光学二区期刊\texttt{Optics Express}——\link{https://doi.org/10.1364/OE.27.024642}{Generation and propagation of a new kind of power-exponent-phase vortex beam}
  \\ 3.光学三区期刊\texttt{Optics Communications}——\link{https://doi.org/10.1016/j.optcom.2017.03.065}{Orbital angular momentum of the vortex beams through a tilted lens}
  \\ 4.提交论文至物理二区期刊\texttt{Annalen der Physik}——{A controllable converter between Hermite Gaussian modes and Laguerre Gaussian modes(submitted)}
\end{itemize}

%======================================================================
\sectionTitle{科研经历}{\faBriefcase}
%======================================================================
\begin{experiences}
  \experience%
    [2018]%
    {至今}%
    {博士研究生}%
    [参加国家自然科学基金项目11874321——基于空间光调制的新型光束产生及其传输特性的研究] 
\end{experiences}
\begin{experiences}
  \experience[2016]{至今}{浙江大学物理系}[负责课题组实验室日常管理和建设]
\end{experiences}
%======================================================================
\sectionTitle{科研技能}{\faCogs}
%======================================================================
\begin{itemize}
  \item 主要研究光学涡旋,熟练掌握涡旋光束的理论和实验进行
  \item 擅长光学传输数值模拟,能够熟练使用MATLAB等主流数值计算软件
  \item 能够独立搭建光路,完成实验,能够实现光学实验自动化操作
  \item 熟练使用SLM以及DMD等光学主流的实验仪器
  \item 能够熟练运用部分相干光理论,以及懂得如何对部分相干光展开实验
\end{itemize}
\sectionTitle{通用技能和语言}{\faWrench}
%======================================================================
\begin{competences}
  \comptence{操作系统}{%
    \icon{\faLinux} Linux (8 年),
    \icon{\faApple} macOS (4 年)
  }
  \comptence{编程}{%
    Python, C, Shell, Matlab, Mathematics
  }
  \comptence{工具}{%
    SSH, Git, Make, Tmux, Vim, \LaTeX
  }
  \comptence{数据分析}{%
	Matplotlib,Matlab,Mathematics
  }
  \comptence{\icon{\faLanguage}语言}{
    \textbf{英语} --- 读写(优良),听说(日常交流)
  }
\end{competences}


\end{document}

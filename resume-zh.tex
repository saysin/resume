% !TEX TS-program = xelatex
%% Copyright (c) 2019-2020 Donghui Shen %%<saysin.github.io>
%% CC BY 4.0 License
%%
%% Created: 2019-10-11
%%

% Chinese version
\documentclass[zh]{resume}

% Adjust icon size (default: same size as the text)
\iconsize{\Large}

% File information shown at the footer of the last page
\fileinfo{%
  \faCopyright{} 2019--2020, Donghui Shen \hspace{0.5em}
  \creativecommons{by}{4.0} \hspace{0.5em}
  \githublink{saysin}{resume} \hspace{0.5em}
  \faEdit{} \today
}

\name{栋辉}{沈}

\keywords{Linux, Programming, Python, C, Shell}

%\tagline{\icon{\faBinoculars}工程师}%寻求岗位
%\tagline{博士研究生}%现在的身份或者职位

\photo{2.5cm}{pic.jpg}

\profile{
  \mobile{132-1610-7661}
  \email{saysin@163.com}
  \github{saysin} \\
  \degree{物理学 \textbullet 博士}
  \university{浙江大学}
  \birthday{1992-12-01}
  \home{浙江 \textbullet 绍兴}
  % Custom information:
  % \icontext{<icon>}{<text>}
  % \iconlink{<icon>}{<link>}{<text>}
}

\begin{document}
\makeheader

%======================================================================
% Summary & Objectives
%======================================================================
{\onehalfspacing\hspace{2em}
物理学专业(光学方向)博士研究生,有扎实的物理、数学与编程基础,
擅长光束传输变换理论模拟和光学实验,热衷计算机和网络技术,
有 8 年的 Linux 和 4年的macOS 使用经验,熟练掌握 Shell、Python 和 C 语言编程。崇尚自由开源的精神,并一直在努力为开源世界贡献自己的力量。
\par}

%======================================================================
\sectionTitle{技能和语言}{\faWrench}
%======================================================================
\begin{competences}
  \comptence{操作系统}{%
    \icon{\faLinux} Linux (8 年),
    \icon{\faApple} macOS (4 年)
  }
  \comptence{编程}{%
    Python, C, Shell, Matlab, Mathematics
  }
  \comptence{工具}{%
    SSH, Git, Make, Tmux, Vim, \LaTeX
  }
  \comptence{数据分析}{%
	Matplotlib,Gnuplot,Matlab,Mathematics
  }
  \comptence{\icon{\faLanguage}语言}{
    \textbf{英语} --- 读写(优良),听说(日常交流)
  }
\end{competences}

%======================================================================
\sectionTitle{教育背景}{\faGraduationCap}
%======================================================================
\begin{educations}
  \education%
    {2017.03}%
    [2020.06]%
    {浙江大学}%
    {理学院物理系}%
    {光学}%
    {博士}

\separator{0.5ex}
\education%
{2015.09}%
[2017.03]%
{浙江大学}%
{理学院物理系}%
{光学}%
{硕士}

  \separator{0.5ex}
  \education%
    {2011.09}%
    [2015.06]%
    {华侨大学}%
    {信息科学与工程学院}%
    {应用物理学}%
    {学士}
\end{educations}

%======================================================================
\sectionTitle{计算机技能}{\faCogs}
%======================================================================
\begin{itemize}
  \item 全国计算机等级考试(C语言)二级合格
  \item 搭建了自己的\link{https://saysin.github.io}{\texttt{blog}}网站
  \item 在谷歌云搭建了自己的服务器
  \item 能够熟练的使用商业软件Matlab和Mathematics进行科研中的数值计算,目前正往Python编程转移
\end{itemize}

%======================================================================
\sectionTitle{个人项目}{\faCode}
%======================================================================
\begin{itemize}
  \item \link{https://github.com/saysin/resume}{\texttt{resume}}:(\LaTeX)此简历的模板和源文件
  \item 完成光学实验室的搭建工作并负责管理实验室
\end{itemize}

%======================================================================
\sectionTitle{科研成果}{\faAtom}
%======================================================================
\begin{itemize}
  \item 发表 2 篇第一作者SCI 论文
  \\ 1.光学Top期刊\texttt{OL}\link{}{}
  \\ 2.光学Top期刊\texttt{OE}

\end{itemize}

%======================================================================
\sectionTitle{实习经历}{\faBriefcase}
%======================================================================
\begin{experiences}
  \experience%
    [2018.04]%
    {2018.08}%
    {数据工程师 @ 上海领脉网络科技(初创公司)}%
    [\begin{itemize}
      \item 从 Amazon 网页搜索并挖取商品与广告信息
        (Python, Requests, BeautifulSoup)
      \item 配置 Airflow 服务器和数据库等基础设施,
        定期从 Amazon 获取产品销售与广告投放等数据
      \item 开发网站(Flask, jQuery),帮助客户优化 Amazon 广告投放
    \end{itemize}]

  \separator{0.5ex}
  \experience%
    [2013.07]%
    {2013.09}%
    {网站开发 @ 97 随访(初创公司)}%
    [\begin{itemize}
      \item 后端开发(Django),完成用户注册、数据存储和搜索等功能
      \item 前端开发(jQuery, AJAX),对患者各项指标随时间的变化进行可视化
    \end{itemize}]
\end{experiences}

\end{document}

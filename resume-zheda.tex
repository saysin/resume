% !TEX TS-program = xelatex
%% Copyright (c) 2019-2020 Donghui Shen %%<saysin.github.io>
%% CC BY 4.0 License
%%
%% Created: 2019-10-11
%%

% Chinese version
\documentclass[zh]{resume}

% Adjust icon size (default: same size as the text)
\iconsize{\Large}

% File information shown at the footer of the last page
\fileinfo{%
  \faCopyright{} 2019--2020, Donghui Shen \hspace{0.5em}
  \creativecommons{by}{4.0} \hspace{0.5em}
  \githublink{saysin}{resume} \hspace{0.5em}
  \faEdit{} \today
}

\name{栋辉}{沈}

\keywords{Linux, Programming, Python, C, Shell}

\tagline{\icon{\faBinoculars}应聘智能网络中心太赫兹团队-博士后}%寻求岗位
%\tagline{博士}%现在的身份或者职位

\photo{3.5cm}{求职.jpg}

\profile{
  \mobile{132-1610-7661}
  \email{shendonghui@zju.edu.cn}
  \github{saysin} \\
  \degree{物理学 \textbullet 博士}
  \university{浙江大学}
  \birthday{1992-12-01}
  \home{浙江 \textbullet 绍兴}
  % Custom information:
  % \icontext{<icon>}{<text>}
  % \iconlink{<icon>}{<link>}{<text>}
}

\begin{document}
\makeheader

%======================================================================
% Summary & Objectives
%======================================================================
{\onehalfspacing\hspace{2em}
物理学专业(光学方向)博士研究生,有扎实的物理、数学与编程基础,
擅长光束传输变换理论模拟和光学实验,主要研究领域为光学涡旋和光场调控。
热爱物理和计算机技术。熟练掌握 Shell、Python 和 C 语言编程。崇尚自由开源的精神,并一直在努力为开源世界贡献自己的力量。
\par}

%======================================================================

%======================================================================
\sectionTitle{教育背景}{\faGraduationCap}
%======================================================================
\begin{educations}
  \education%
    {2015.09}%
    [2020.06]%
    {浙江大学}%
    {理学院物理系}%
    {光学}%
    {博士(硕博连读)}

\separator{0.5ex}
\education%
{2011.09}%
[2015.06]%
{华侨大学}%
{信息科学与工程学院}%
{应用物理学}%
{学士}

\separator{0.5ex}
\education%
{2008.09}%
[2011.06]%
{春晖中学}%
{理科}%
{理科}%
{高中生}
\end{educations}

%======================================================================
\sectionTitle{科研成果}{\faAtom}
%======================================================================
研究生期间发表2篇第一作者SCI论文,1篇第三作者SCI论文。另有三篇本科期间所在课题组共同发表的论文
\begin{itemize}
  \item 1.光学二区期刊\texttt{Optics Letters}(\textbf{Editor's Pick})——\link{https://doi.org/10.1364/OL.44.002334}{\textbf{Donghui Shen} and Daomu Zhao, "Measuring the topological charge of optical vortices with a twisting phase," Opt. Lett. \textbf{44}, 2334-2337 (2019)}
  \item 2.光学二区期刊\texttt{Optics Express}——\link{https://doi.org/10.1364/OE.27.024642}{\textbf{Donghui Shen}, Ke Wang, and Daomu Zhao, "Generation and propagation of a new kind of power-exponent-phase vortex beam," Opt. Express \textbf{27}, 24642-24653 (2019)}
  \item 3.光学三区期刊\texttt{Optics Communications}——\link{https://doi.org/10.1016/j.optcom.2017.03.065}{Meilan Luo, Zhaohui Zhang, \textbf{Donghui Shen} and Daomu Zhao, "Orbital angular momentum of the vortex beams through a tilted lens," Opt. Commun. \textbf{396}, 206-209(2017)}
  \item 4.朱清智,\textbf{沈栋辉},吴逢铁,何西.部分相干光对周期性局域空心光束的影响[J].物理学报,2016,65(04):71-78.
  \item 5.何艳林,\textbf{沈栋辉},孙川,吴逢铁.增大Bessel光束无衍射距离的凹透镜系统[J].红外与激光工程,2016,45(05):152-157.
  \item 6.陈姿言,吴逢铁,何西,\textbf{沈栋辉}.体视显微镜与CCD成像系统在高阶贝塞尔光束光斑测量中的应用[J].强激光与粒子束,2014,26(11):33-38.
\end{itemize}

%======================================================================
\sectionTitle{荣誉奖项}{\faAtom}
%======================================================================
\begin{itemize}
  \item 2018-2019获得教育部国家奖学金
  \item 2018-2019获得浙江大学三好研究生、优秀研究生荣誉称号
  \item 2017-2018获得浙江大学优秀研究生荣誉称号
  \item 2015-2016获得浙江大学优秀学生干部、优秀研究生荣誉称号
  \item 2013-2014获得教育部国家奖学金
  \item 2012-2013获得华侨大学校一等奖学金
  \item 2011-2012获得教育部国家奖学金
  
\end{itemize}
\newpage
\sectionTitle{科研技能}{\faCogs}
%======================================================================
\begin{itemize}
  \item 主要研究光学涡旋,熟练掌握涡旋光束的理论和实验进行
  \item 擅长光学传输数值模拟,能够熟练使用MATLAB等主流数值计算软件
  \item 能够独立搭建光路,完成实验,能够实现光学实验自动化操作
  \item 熟练使用SLM以及DMD等光学主流的实验仪器
\end{itemize}


%======================================================================
\sectionTitle{科研经历与社会工作}{\faBriefcase}
%======================================================================
\begin{experiences}
  \experience[2020.7]{至今}{浙江理工大学讲师}[光场调控省重点实验室,陈瑞品教授课题组]
\end{experiences}
\begin{experiences}
  \experience%
    [2018.12]%
    {2020.6}%
    {博士研究生}%
    [参加国家自然科学基金项目11874321——基于空间光调制的新型光束产生及其传输特性的研究] 
\end{experiences}
\begin{experiences}
  \experience[2015.9]{2020.6}{浙江大学物理系赵道木教授课题组}[负责课题组实验室日常管理和建设]
\end{experiences}  
\begin{experiences}
  \experience[2016.9]{2017.9}{浙江大学物理系}[担任光学与无线电研究生党支部书记]
\end{experiences}
\begin{experiences}
  \experience[2014.10]{2016.10}{华侨大学信息科学与工程学院}[2014年国家级大学生创新创业训练计划——LED光源在特殊光束中的应用,担任负责人]
\end{experiences}
%======================================================================

\sectionTitle{通用技能和语言}{\faWrench}
%======================================================================
\begin{competences}
  \comptence{操作系统}{%
    \icon{\faLinux} Linux (8 年),
    \icon{\faApple} macOS (4 年)
  }
  \comptence{编程}{%
    Matlab, Mathematica,Python, C, Shell, 
  }
  \comptence{工具}{%
    SSH, Git, Make, Tmux, Vim, \LaTeX
  }
  \comptence{数据分析}{%
	Matlab,Origin,Matplotlib
  }
  \comptence{\icon{\faLanguage}语言}{
    \textbf{英语} --- 读写(优良),听说(日常交流);大学英语六级(523)
  }
\end{competences}

%======================================================================
\sectionTitle{自我评价}{\faAtom}
%======================================================================
为人诚实可靠又不失灵活变通,遇事冷静,不盲目冲动,对于犯下的错误能够主动反思,不逃避。对待
学习和工作认真踏实,做事有条理性,能够承受压力,并化压力为动力继续前行。熟悉光学科研领
域的前沿工作,能够独立完成科研工作,也能与他人开展合作,沟通能力强。
\end{document}
